\documentclass[12pt,letterpaper]{article}
\usepackage{graphics}
\addtolength{\textwidth}{2cm}
\addtolength{\oddsidemargin}{-1cm}
\addtolength{\textheight}{4cm}
\addtolength{\topmargin}{-2cm}

\begin{document}
\author{Alexander Wagner}
\title{The graph library:\\simple graphical steering for scientific
  applications }
\maketitle
\newpage
\tableofcontents
\newpage
\section{History}
In my first year as a physics student at Bielefeld
University in 1992, I met Johannes Schlosser. What began as a study group
for physics problems soon developed into a great friendship and an
intense and fruitful collaboration on exploring what computers could
do.

At this time a computer would be in ``text mode'' as standard and only
for special problems would a ``graphics mode'' be activated. At this
time only Apple computers broke with this paradigm and Microsoft was
only a small company making the Microsoft DOS as well as a few text
programs. A graphics resolution of 600:300 was a great resolution and
color was a luxury we could not afford. We each bought an 8068 IBM
compatible computer and began checking out what we could do with
it. Switching the monochrome Hercules graphics card into graphics mode
was a major challenge. We started using Turbo Pascal as our
programming language of choice with a few Assembler inline additions.

But seeing the graphics on the screen is rather fleeting, so we
also wanted to be able to print it. There was the option of a screen
print, but that meant limiting the print output to the low resolution
of the screen, even though I had invested in an NEC P6c, a color dot
printer. So we wanted to make sure that we could print out our
graphics with maximal resolution and in color on our printer. But how
can you transfer a picture from the screen to the printer and increase
the resolution? The solution we found was to have two sets of basic
graphic routines: one would write to the screen, the other would write
to a set of memory which was a bitmap of what we would then write to
the printer. We realized that if you wrote your program so that all
sizes would be relative to the X and Y dimensions of your medium, they
would look right, no matter if they would be printed on a screen or
printed to the printer.

On this level, we managed to program a graphical user interface which
was supposed to be easy to adapt for new programs. If you wanted to
print something you could use the same routines that you had used to create
your graphics on the screen. It came as a total revelation to us when,
a few years later, Peter Seroka introduced us to the X windows system
common on Unix workstations. Being able to open windows on other
computers seemed like a little miracle to us. A little earlier Linus
Torwald had developed the Linux kernel and now the first working
versions of the Linux operating systems became available and they were
free! Another revelation there. So with the help of Linux (version
0.92, I believe) we could transform our computers (80286 IBM
compatibles at this time) into Unix workstations. 

This signaled the change from Pascal to C (which only our friend Tim
Scheideler was using at the time) and X windows for our graphics
needs. Previously all printing was different for all printers, but the
people of Adobe Inc. had developed PostScript, a portable graphics
language. And with the help of ghostscript you could (and still can)
print any postscript file on any printer. So we decided to now use
PostScript files as the output which could then be printed
\textit{everywhere}. 

At this time both Johannes and myself were working on our
Diplomarbeit (masters thesis) at Bielefeld University. Not
surprisingly, we were both doing computational projects and were
heavily relying on graphical output. At this point we had
actually developed two separate graphics libraries. Johannes had true
three-dimensional capabilities whereas I had developed basic menus,
the printer support, and coordinate systems. Because Johannes was the
much more organized of the two of us we used the structure of this
graphics library and added my extensions into his code to create what
remains to this day the basic foundation of the library: a set of
basic two dimensional graphics routines which interface either xlib of
the X windows system or PostScript routines, a library for the
representation of three dimensional scenarios. This was supported by a
set of menu functions that dealt with mouse interactions.

After we successfully used the graphics libraries for our Diplomarbeit
Johannes went on to a rapidly advancing career at IBM Deutschland
whereas I went on to graduate school at Oxford. For my graduate work
I picked a numerical project again, but this time I dealt with complex
fluids which needed a number of different representations that our
graphics library did not yet support. I included support for density
plots and contour lines, but whenever I wrote a new program I still
copied lots of old code from the last programs to reuse them in the
new programs. And then I realized that anytime I copied code from other
programs, it meant that it should be part of the library and not part
of the user code. It took quite a few numbers of iterations to reduce
the amount of code the user had to write to use the graphics code. 

In 1998, at the end of my D.Phil.~(Ph.D.), I considered briefly
founding a company for simulating complex fluids and using the
graphics as a part of this commercial enterprise. But I soon realized
that my heart was with science and I did not enjoy the prospects of
meeting with venture capitalists to fund my software company (and that
was at the beginning of the Internet bubble). Instead, I continued my
academic career with postdoctoral positions at MIT and Edinburgh
University. At each of these places I realized that scientists as a
whole don't tent to have easy access to the data in their computer
simulations and that the approach I had taken was quite unique.

Now I have a faculty position at North Dakota State University and I
want to give you the ability to steer your computer simulations and to
monitor them closely. Considering how much Johannes and I have
benefited from the free software project we have decided to release
our library under the GNU public license to make sure that it is
available for free to anyone.

\section{So what does it do?}
The project began as a library for real time three-dimensional
visualization. We had video games in mind, as well as scientific
visualization. In the end we never finished the video game, but the
scientific visualization part became more and more important. From
these beginnings a a library developed that is supposed to make
scientific visualization as easy as possible. In this introduction I
will leave out the details of the graphics machine and focus only on
the use of library functions for scientific visualization.

I hope that a full documentation of the software will become available
in the future to make it easier for others to extend the library.

\section{Scientific Visualization}
Imagine that you want to simulate a fluid mixture that is
phase-separating. For scientific reasons you may be interested in some
statistical properties of this process, but how can you be sure that
the program is really doing what you expect it to do? And what do you
do if you want to find out what the effect of different parameters is?
There will often be quite a few different parameters that you need to
control. 

My vision for this library is that it will help scientists to
transform their simulations which are performing their instruction in
a way that is hard to observe, into a transparent object where you can
look at all times at the relevant variables and where you can change
the parameters to see what will happen. This changes the nature of
computational research: suddenly you are able to see and manipulate
the system you simulate directly. You will now be able to see what is
going on in the system, and if something does behave differently than
you expect you have probably found a bug in your program and you can
eliminate it faster. And if it isn't a bug you have just learned
something interesting about the system. And once in a while you will
be the first to notice and you have made a discovery that you might
not have made without being able to observe the system so
closely (this is what happened to me a lot of the time).

This is not to suggest that you give up the more traditional approach
of defining curves for which you can make analytical predictions or
which you can compare to experiments. But I feel that it is a very
valuable extension of this approach.

\section{Examples}
Say you want to write a code that solves the one-dimensional diffusion
equation with periodic boundary conditions. Don't worry if you don't
understand this particular problem. It is just a specific example of a
computational problem that you might have. Understanding the details
of this is not important in order to follow the main points of this
section. The diffusion equation is given by
\begin{equation}
\frac{\partial \rho}{\partial t} =
\kappa\frac{\partial^2\rho}{(\partial x)^2}
\end{equation}
You can model this equation by writing a small lattice Boltzmann
program. You could use many other ways, but lattice
Boltzmann is close to my heart. If you are interested in the details
of this algorithm you can find an explanation in appendix
\ref{1d_diff_LB}.  Your program\footnote{This program is included as
\texttt{prog1.c} in this distribution.} might look something like
this:
\begin{verbatim}
#include <stdio.h>
#include <math.h>

#define size 100
static Repeat=1000;
static double f0[size],f1[size],f2[size], omega=1, T=0.3;

void init(){
  int i;
  double density;
  for (i=0;i<size;i++){
    density=(2+sin(2*M_PI*i/size));
    f0[i]=density*(1-T);
    f1[i]=density*T*0.5;
    f2[i]=density*T*0.5;
  }
}

void iterate(){
  int i;
  double density,tmp1,tmp2;
  for (i=0;i<size;i++){
    density=f0[i]+f1[i]+f2[i];
    f0[i]+=omega*(density*(1-T)-f0[i]);
    f1[i]+=omega*(density*T*0.5-f1[i]);
    f2[i]+=omega*(density*T*0.5-f2[i]);
  }
  tmp1=f1[size-1];
  tmp2=f2[0];
  for (i=1;i<size;i++){
    f1[size-i]=f1[size-1-i];
    f2[i-1]=f2[i];
  }
  f1[0]=tmp1;
  f2[size-1]=tmp2;
}

int main(){
  int i;
  init();
  for (i=0;i<Repeat;i++) iterate();
  return 0;
}
\end{verbatim}
This is a very simple program which currently only consists of the
main computational kernel. It does not ``do'' anything yet, since
it has no output. It has been my general experience that the actual
computation can usually be written in a succinct bit of code whereas
the analysis and output make up the main part of the code. And this
analysis tends to be so individual that there is little guidance in
how to write this bit of the code. 

Now we want to see what the code actually does. So we would like to
see what the density is at each new time step and how it evolves. We
might want to be able to re-initialize the simulation and maybe to
initialize it with different initial conditions. Also, we might want to
be able to change the size of the simulation. And on the fly, we might
want to be able to change the diffusion constant which in this
simulation is $\kappa=(1/\omega-0.5)T$. So we might want to change the
two parameters independently so that we can examine how the numerical
errors differ for the two approaches.

This may sound like quite a laborious task, but this is exactly what the
mygraph library was designed to do. What we have to do is to tell the
library which data we want to look at, give it some hints about its
size, and it will be able to display it appropriately. We will also
need a few other control variables to be able to tell the program that
it should pause, run for only one step at a time, or that it should
re-initialize. So this program, with a full graphical user interface
(GUI),\footnote{This program is distributed as \texttt{prog2.c} with
  this distribution.} would look like this:  
\begin{verbatim}
#include <stdio.h>
#include <math.h>
#include <mygraph.h>

#define SIZE 100
static int size=SIZE,Repeat=1000,done=0,sstep=0,pause=1;
static double f0[SIZE],f1[SIZE],f2[SIZE], omega=1, T=0.3,Amplitude=1;
static double density[SIZE];
static int densityreq=0;


void init(){
  int i;
  for (i=0;i<size;i++){
    density[i]=(2+Amplitude*sin(2*M_PI*i/size));
    f0[i]=density[i]*(1-T);
    f1[i]=density[i]*T*0.5;
    f2[i]=density[i]*T*0.5;
  }
}

void init2(){
  int i;
  for (i=0;i<size;i++){
    if (2*i>=size) density[i]=2+Amplitude; else density[i]=2-Amplitude;
    f0[i]=density[i]*(1-T);
    f1[i]=density[i]*T*0.5;
    f2[i]=density[i]*T*0.5;
  }
}

void iterate(){
  int i;
  double tmp1,tmp2;
  for (i=0;i<size;i++){
    density[i]=f0[i]+f1[i]+f2[i];
    f0[i]+=omega*(density[i]*(1-T)-f0[i]);
    f1[i]+=omega*(density[i]*T*0.5-f1[i]);
    f2[i]+=omega*(density[i]*T*0.5-f2[i]);
  }
  tmp1=f1[size-1];
  tmp2=f2[0];
  for (i=1;i<size;i++){
    f1[size-i]=f1[size-i-1];
    f2[i-1]=f2[i];
  }
  f1[0]=tmp1;
  f2[size-1]=tmp2;
}


void GUI(){
  DefineGraphN_R("Density",density,&size,&densityreq);
  StartMenu("GUI",1);
    DefineDouble("T",&T);
    DefineDouble("omega",&omega);
    StartMenu("Restart",0);
      DefineMod("size",&size,SIZE+1);
      DefineDouble("Amplitude",&Amplitude);
      DefineFunction("Restart sin",&init);
      DefineFunction("Restart step",&init2);
    EndMenu();
    DefineGraph(curve2d_,"Density graph");
    DefineBool("Pause",&pause);
    DefineBool("Single step",&sstep);
    DefineInt("Repeat",&Repeat);    DefineBool("Done",&done);
  EndMenu();
}

int main(){
  int i;
  init();
  GUI();
  while (!done){
    Events(1); /* Whenever there are new data the argument of 
		  Events() should be nonzero. This will set the
		  requests for data so that you can calculate them
		  on demand only. For this simple program you can
		  always set it to one. */
    DrawGraphs();
    if (!pause || sstep){
      sstep=0;
      for (i=0;i<Repeat;i++) iterate();
    } else {
      sleep(1);/*when the program is waiting it returns the 
		 CPU time to the operating system */
    }
  }
  return 0;
}
\end{verbatim}
Let us go briefly through the new additions to the program which
implement the GUI. The function \texttt{GUI()} tells the graphical
user interface about the data to display and the variables we want to
be able to change interactively. Firstly we have the one-dimensional
density field that we want to display. This data gives for a
section of the natural numbers \texttt{N} one real number
\texttt{R}. The function that tells the GUI about this is
\begin{verbatim}
  DefineGraphN_R("Density",density,&size,&densityreq);
\end{verbatim}
The first argument is a string which corresponds to the name of the
data as it will appear in the menus. The second argument is the data
you want to display. More exactly, it is a pointer to the first
element in the array. The third argument is a pointer to a variable
giving the size of the data. We use a pointer rather than simply a
number because if the size of the data changes during the simulation (and
this change is reflected in a change of the variable \texttt{size})
the GUI will know about this and display the data
correctly.
Also note that we made the density an array, not just a temporary
variable of the iterate routine, so that we can display it.

Then we start the menu for the GUI.
\begin{verbatim}
  StartMenu("GUI",1);
\end{verbatim}
The first argument is the name of the menu, and the second argument is
either 0 or 1. If it is 1 this indicates that the menu will be
initially displayed. Since this is the first menu, it should certainly
be displayed. Next we define two \texttt{double} variables as menu
items
\begin{verbatim} 
    DefineDouble("T",&T);
    DefineDouble("omega",&omega);
\end{verbatim}
These functions have two arguments: the name as it appears in the menu
and the address of the variable. We now define a sub-menu with a new
\texttt{StartMenu()} function as above.
\begin{verbatim}
    StartMenu("Restart",0);
      DefineMod("size",&size,SIZE+1);
      DefineDouble("Amplitude",&Amplitude);
      DefineFunction("Restart sin",&init);
      DefineFunction("Restart step",&init2);
    EndMenu();
\end{verbatim}
There are three new routines: 
\begin{description}
\item{\texttt{DefineMod("size",\&size,SIZE+1)}} inserts a menu item for
  the variable size and only allows it to vary from \texttt{0} to
  \texttt{SIZE}. 
\item{\texttt{DefineFunction()}} allows you to
start functions which don't have any arguments to be called at the
press of a button. The first argument of this function is again the
name as it appears in the menu and the second is the address of the
function you want to be able to call. Note that we introduced a second
initialization routine that initializes the density as a step function
to make things a bit more interesting.
\item{\texttt{EndMenu()}} tells the GUI that this is the end of the
sub-menu.
\end{description}
Next we have a special menu item to display line graphs.
\begin{verbatim}
    DefineGraph(curve2d_,"Density");
\end{verbatim}
The routine that does this is \texttt{DefineGraph()}. The first
argument of this function is an integer that is represented by the name
\texttt{curve2d\_}.\footnote{There are several other types corresponding
  to two-dimensional density and vector plots, three-dimensional graph
  representations, three-dimensional contour plots, etc. This is
  discussed in section \ref{graphs}.} 
The rest of the GUI implementation should now be self-explanatory.
\begin{verbatim}
    DefineBool("Pause",&pause);
    DefineBool("Single step",&sstep);
    DefineInt("Repeat",&Repeat);    
    DefineBool("Done",&done);
  EndMenu();
\end{verbatim}

After we have successfully initialized the GUI there are two more
library functions that you have to be aware of. Firstly
\begin{verbatim}
    Events(1); 
\end{verbatim}
For any program with a graphical user interface you have to give the
library functions a chance to react to mouse and keyboard events. It
is this routine that does it. So you have to make sure that you call 
\texttt{Events()}\footnote{Whenever there are new data the argument of 
          \texttt{Events()} should be nonzero. This will set the
          requests for data so that you can calculate them
          on demand only. For this simple program you can
          always set it to one.}
regularly so that windows can be redrawn, resized and mouse clicks can
be acted upon. The second routine is 
\begin{verbatim}
    DrawGraphs();
\end{verbatim}
This routine will draw all the graphs that have need to be
displayed.\footnote{But why is this not simply part of the \texttt{Events()}
function? The reason lies in the following consideration: often there
are quantities that you want to monitor, but that are not actually
required for the computation. Having two routines allows you to
calculate only when they are needed. This is the reason that each data
is associated with a request flag. If the user chose to observe this
data, the request flag would be set in the \texttt{Events()}
routine and you could write a routine that would calculate this data
before you called the \texttt{DrawGraphs();} routine.}
One other consideration to keep in mind is that interacting with a
user is slow. So you really don't want to call the \texttt{Events()}
routine too often because that would make your program slow.

Now I just want to make a few more remarks about some useful steering
parameters that I tend to use for simulations. We want to be able to
stop the calculations temporarily to look at the data in leisure,
to be able to step through the calculation one iteration at the time and
to re-initialize the calculation. And we want to be able to quit the
computation at the press of a button. So we define the variables
\begin{center}
\begin{tabular}{|l|p{8cm}|}
\hline
variable & description \\
\hline
\texttt{done} & this is equal to 0 until the simulation is finished\\
\texttt{pause}& this is equal to 0 unless the simulation is paused\\
\texttt{sstep}& this variable gets set to one to run the simulation
for a single step\\
\texttt{Repeat}& this variable is set to the number of steps that the
simulation should be run until the \texttt{Events()} function is
called again\\
\hline
\end{tabular}
\end{center}
With this explanation of the variables the logic of the main routine
should be obvious.
\begin{verbatim}
  while (!done){
    Events(1);
    DrawGraphs();
    if (!pause || sstep){
      sstep=0;
      for (i=0;i<Repeat;i++) iterate();
    } else {
      sleep(1);/*when the program is waiting it returns the 
                 CPU time to the operating system */
    }
  }
\end{verbatim}
The only other comment I want to make is regarding the
\texttt{sleep(1)} statement. This routine takes control away from the
program and gives it to the operating system. Why would you want to do
this? This is mainly a question of consideration for others and for
any of your own programs that might be running at the same
time. Without this statement the program would always use all the CPU
cycles it can get its hands on, even if it is doing nothing else than
waiting for any input you might want to give it. And while it is
waiting, it might as well give some time to programs that really
need it.

\subsection{Compiling the program}
In order to compile the program you have to tell it where to find the
header files for the simulation and where to find the library. Then
you need to link your program to the graph and X11 libraries. Assuming
that you use the standard implementation (see \ref{install} Installing
the library), the command will look something like
\begin{verbatim}
cc -I ~/include prog2.c -L ~/lib -L /usr/X11R6/lib -lm -lgraph 
-lX11 -o anim2.out
\end{verbatim}
This assumes that you have installed the library in your home
directory. For system-wide installations you would not need the special
\texttt{-I ~/include} and \texttt{-L ~/lib} options. For more
explanations see the man pages of your c compiler.

\subsection{Running the program}

\begin{figure}
\begin{center}
\resizebox{0.8\textwidth}{!}{\includegraphics{screen1.eps}}
\end{center}
\caption{Screen-shot of the GUI after a graphics window has been opened
and the density graph has been selected. If you right-click on the
density option you can choose symbols and colors for the graph.}
\label{fig1}
\end{figure}

If you now type 
\begin{verbatim}
./prog2
\end{verbatim}
you will see the GUI appear. The GUI is shown in Figure
\ref{fig1}. First you may want to click with the left mouse button on
the ``Density graph'' button to see the density. Then in the density
graph you may want to select the ``Density'' with the left mouse
button. If you click with the right mouse button on the ``Density''
field you can choose several properties of the graph. If you want to
remove the new menu simply right-click the ``Density'' button as
well. This is a general method to remove menus that you have
created. The only exceptions are graphics windows because you can
create an unlimited number of them.

Now clicking on the ``Control'' button will give you a menu that
allows you to choose several properties of the graph. You may want to
toggle the ``AutoScaling'' button. Next right-click on the ``Repeat = 1000''
button in the main menu. A cursor will appear and you can change this
value to 1. Make sure that you finish the input by pressing enter. The
program will not allow you to do anything else before you do this.

Now, if you are ready to look at the evolution of the density, press
the ``Pause'' button and watch the evolution of the density. You can
see that the initial density fluctuation decays and leaves us with a
homogeneous solution. We actually know that analytically the sin
function with an exponentially decaying amplitude is a solution of the
diffusion equation. So now press ``AutoScaling'' to on and
press the Restart button. In the new menu press ``Restart now''. Now
you can see the evolution of the density and the form indeed remains
constant (except for the moving scale) so that we see that our
numerical approach is in agreement with the analytical solution.

To finish the program simply press the ``done'' button.

\section{Reference section}
\label{graphs}
In this section you can find the list of all the user commands (and
some functions that you are less likely to need) that you are going to
need to visualize and steer you programs. 

In table \ref{graphcommands} you find functions that tell the GUI what kind of data is available to be visualized. The general notation contains Nx...xN\_Rx...xR where the number of N indicate the dimensionality of the data array and the number of R indicate the dimensionality of the vector output. While there are a large number of possible combination, only a few are easily visualized, and special visualizations are implemented for each data type.

The second part of this table contains some default values that you can associate with each data type defined after calling these default functions. Currently these apply only to data types represented by curve2d\_. The options that you can use here are listed in table \ref{ShapeAndColor}.

Lastly there is the possibility of separating out different graph sections. This can be useful if you have too many data to be represented in the menu of one graph window. In these cases you can define new graphs structures, and the graphics windows refere to the data given by the current active graph. 

%If you want to display particles in a two dimensional geometry you can
%do so by first setting \texttt{SetDefaultLineType(NoLine)},
%\texttt{SetDefaultShapede(Circle)} and \texttt{SetDefaultFill(1)} and
%then defining them as \texttt{N\_RxR} data types. You can then use the
%\texttt{curve2d\_} graphical representation to display them.

\begin{table}
\begin{center}
\begin{tabular}{|p{0.6\textwidth}|p{0.35\textwidth}|}
\hline
function & description\\
\hline
\texttt{DefineGraphN\_R(char *Name,double *gd,int *dim,int *req)}&
Simple one dimensional arrays\\
\texttt{DefineGraphN\_RxR(char *Name,double *gd,int *dim,int *req)}&
One dimensional arrays of points (x,y).\\
\texttt{DefineGraphN\_RxRp(char *Name,double **gd,int *dim,int *req)}&
Pointer to an one dimensional array of points. Needed if the size of
the data field will be resized with \texttt{realloc()}.\\
\texttt{DefineGraphNxN\_R(char *Name,double *gd,
                      int *dim1,int *dim2,int *req)}& Two dimensional array.\\
\texttt{DefineGraphNxN\_RxR(char *Name,double *gd,
                      int *dim1,int *dim2,int *req)}& Two dimensional
array of vectors.\\
\texttt{DefineGraphNxN\_RxRxRt(char *Name,double *gd,
                      int *dim1,int *dim2,int *req)}& Three
dimensional field. \textit{Still waiting for the implementation of the
 3 dimensional contour.}\\
\texttt{DefineGraphContour3d(char *Name,Contour **gd,
                      int *dim1,int *dim2,int *dim3,int *req)}& A
special data type to define contour, especially from a parallel program.\\
&In the DefineGraphXXX routines the last argument \texttt{*req} can be
a \texttt{NULL} pointer if the data do not need to be recalculated.\\
\hline
\texttt{SetDefaultColor(int c)}&Default color for graphs
defined after this command. See Table \ref{ShapeAndColor}.\\
\texttt{SetDefaultLineType(int s)}& Default line type is 1 (solid
line). 0 means no line. \\
\texttt{SetDefaultShape(int s)}& Default shape according to table 
\ref{ShapeAndColor}.\\
\texttt{SetDefaultSize(double s)}& Default size is one.\\
\texttt{SetDefaultFill(int f)}& Default fill is off (0).\\
\hline
\texttt{NewGraph()}& Start a new set of graphs.\\
\texttt{SetActiveGraph(int)}& Set this before you call 
\texttt{DefineGraph()}.\\
\hline
\end{tabular}
\end{center}
\caption{All commands used to register data that you want to
  visualize}
\label{graphcommands}
\end{table}

\begin{table}
\begin{center}
\begin{tabular}{|l|c|}
\hline
Name & value\\
\hline
NoShape      &0\\
UpTriangle   &1\\
DownTriangle &2\\
Square       &3\\
Circle       &4\\
\hline
\end{tabular}\hspace{5ex}
\begin{tabular}{|l|l|}
\hline
Color functions& English\\
\hline
schwarz() & black \\
weiss()   & white \\
rot()     & red   \\
gruen()   & green \\
blau()    & blue  \\
gelb()    & yellow\\
\hline
\end{tabular}
\end{center}
\caption{Shapes and standard colors. \textit{Colors will be translated soon.}}
\label{ShapeAndColor}
\end{table}


\begin{table}
\begin{center}
\small
\begin{tabular}{|l|p{5cm}|}
\hline
Function & description\\
\hline
\texttt{StartMenu(char *Name,int installed)}& Start a new menu with
name \textit{Name}. It will be visible as you start the program if
\textit{installed} is 1 \\
\texttt{StartBoolMenu(char *Name,int *b)} & This will also start a
menu, but the menu is also a Boolean field (see below) with name
\textit{Name} and address \textit{*b}. To change the value of
\textit{b} left click the button, to open the menu right click the button.\\
\texttt{EndMenu()}&Each menu has to be terminated by an
\texttt{EndMenu()} command.\\
\hline
\texttt{DefineInt(char *Name,int *myint)}& Menu item of type int\\
\texttt{DefineLong(char *Name,long *myint)}& Menu item of type long \\
\texttt{DefineMod(char *Name,int *myint,int mod)}& Menu item of type
int. The value is constraint to be between \texttt{0} and \texttt{mod-1}\\
\texttt{DefineBool(char *Name,int *myint)}& Integer that can only take
the values 0 and 1. In the menu this is represented as ``off'' and ``on''
respectively. \\
\texttt{DefineDouble(char *Name,double *mydouble)}& Menu item of type double\\
\texttt{DefineFloat(char *Name,float *myfloat)}& Menu item of type float\\
\texttt{DefineString(char *Name,char *s,int maxlen)}&Menu item
for a string with maximum length maxlen. \textit{There is a bug that
  still allows you to write strings longer than maxlen}. \\
\hline
\texttt{DefineFunction(char *Name,fp myfp)}& This function allows you
to call a function. This function may not have an argument.\\ 
\texttt{DefineItem(int vartype)}& This is a function to add special
items like the print button (\texttt{print\_} and \texttt{dopint\_}),
a close button (\texttt{close\_}) or a Redraw button
(\texttt{newdraw\_}). You will probably not need to use these except
  maybe the \texttt{close\_} option.\\
\texttt{DefineGraph(int vartype, char *Name)}& Adding buttons for
different kinds of graphics with arguments given by table \ref{graphtable}.\\
\hline
\end{tabular}
\end{center}
\caption{Menu commands}
\end{table}

\begin{table}
\begin{center}
\begin{tabular}{|l|c|p{7cm}|}
\hline
Type  & numerical value & description\\
\hline
\texttt{curve2d\_} & 11  & displays \texttt{N\_R,N\_RxR}, and
                          \texttt{N\_RxRp} data types as two
			  dimensional curves.\\
\texttt{contour2d\_} & 9 & displays \texttt{NxN\_R} data types as
                          density and contour plots and
			  \texttt{NxN\_RxR} data types as vector plots.\\
\texttt{graph2d\_}  & 8  & displays \texttt{NxN\_R} data types as three
                          dimensional graphs.\\ 
\texttt{contour3d\_} & 10& displays \texttt{NxNxN\_R} data types as
                          three dimensional contour plots. \textit{Not
			    yet fully implemented.} \\
\texttt{freedraw\_} & 25 & Displaying the output of graphics routines you have written yourself, and have defined through AddFreedraw() (see Table \ref{freedraw}).\\                          
\hline
\end{tabular}
\end{center}
\label{graphtable}
\caption{Different options for the \texttt{DefineGraph()} routine.}
\end{table}

\begin{table}
  \begin{tabular}{|p{0.48\textwidth}|p{0.48\textwidth}|}
    \hline
    function & description\\
    \hline
    \texttt{AddFreedraw(char *name, FrDr FreeDraw)} & Define a drawing function that you defined yourself of type \texttt{void Name(int xdim,int ydim)}. The size of the window you are drawing in is xdim,ydim. Note that these values change, as the user resizes the window.\\
      \hline
      \texttt{mydrawline(int color, int x1, int y1, int x2, int y2)} & Draws a line between the points $(x_1,y_1)$ and $(x_2,y_2)$. \\
      \texttt{mypolygon(int color, XPoint *points, int n)}&Draws a filled polygon of n XPoints\\
      \texttt{mypolygon\_line(int col, int colLine, XPoint *points, int n)}& Drasw a filled polygon of color col with an outline of color colLine.\\
      \texttt{mylinepolygon(int col, XPoints *points, int n)}& As above, but only draws the outline.\\
        \texttt{mycircle(int col, int x, int y, int r)}& Draws a circle around the point (x,y) with radius r in color col.\\
        \texttt{myfilledcircle(int col, int x, int y, int r)}& Draws a filled circle around the point (x,y) with radius r in color col.\\
        \texttt{mytext(int col, int x, int y, char *text, int orient)}& Draws the string \texttt{text} at position x with orientation 1:centered, 2:right, or left for all other values.\\
      \texttt{myselectfont(int font, char *testtext, double length)} & \textbf{Has to be called before the first \texttt{mytext} command!} It selects a font such that the string will have approximately the length \texttt{length}. The first variable selects the kind of font, and should be set to 0 for now.\\
      \hline
  \end{tabular}
  \caption{Commands used for freedraw graphics routines.}
  \label{freedraw}
\end{table}

\section{Does it work with other programming languages?}
You can use this library on any system supporting the X windows system
and able to link c-libraries. I have used it successfully within the
parallel programming language ZPL.

\appendix

\section{Installing the library}
\label{install}
First you need to obtain the \texttt{graph.tar.gz} file either from
the web at https://www.ndsu.edu/pubweb/~carswagn/
or from another source and put it in your home directory.
Also make sure that you have installed the development libraries for the X11 system.
You then type
\begin{verbatim}
tar xzvf graph.tar.gz
cd c/graph
make install
\end{verbatim}
If you don't get error messages this will successfully install the
library in your system.

\end{document}
